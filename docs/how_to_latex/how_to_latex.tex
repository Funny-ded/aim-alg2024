\documentclass{article}
\usepackage[utf8]{inputenc}

\usepackage{amssymb}

\usepackage[T2A]{fontenc}
\usepackage[english, russian]{babel}

\usepackage[colorlinks = true,
            linkcolor = blue,
            urlcolor  = blue,
            citecolor = blue,
            anchorcolor = blue]{hyperref}

\usepackage{parskip}

\title{Как техать без регистрации и SMS}
\author{elijah}
\date{}

\begin{document}

\maketitle

\section*{Секция}

Можно писать обычным текстом, можно \textit{курсивом}, можно \textbf{жирным}.



\section*{Ненумерованная секция}

Подпункты в тексте можно оформлять так:

\subsection{Подсекция}

Чтобы отделить друг от друга две части решения, можно использовать

маленький отступ:

\smallskip

средний отступ:

\medskip

большой отступ:

\bigskip

Кликабельную ссылку в тексте можно дать так:
\href{https://arxiv.org/pdf/1406.7444.pdf}{эта} статья.

В преамбулу .tex файла нужно включить пакет hyperref, как в начале этого файла.

Нумерованный список можно задать так:

\begin{enumerate}
    \item первый пункт
    \item второй пункт
\end{enumerate}

Ненумерованный так:

\begin{itemize}
    \item первый пункт
    \item второй пункт
\end{itemize}

Оба типа списков поддерживают вложенность, то есть можно написать

\begin{enumerate}
    \item первый пункт
    \begin{enumerate}
        \item первый подпункт
        \item второй подпункт
    \end{enumerate}

    \item второй пункт
\end{enumerate}

Что касается формул, самые важные для нас следующие:

Красивые значки для множеств чисел $\mathbb{R}$, $\mathbb{N}$ (требуют подключения пакета amssymb, как в начале этого файла)

Значок для эпсилон в привычном начертании: $\varepsilon$

Дроби $\frac{A}{B}$

Суммы $\sum\limits_{i=1}^N 2^i$

\end{document}
