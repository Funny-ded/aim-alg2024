\documentclass{article}
\usepackage[T2A]{fontenc}
\usepackage[utf8]{inputenc}   
\usepackage[english, russian]{babel}

% Set page size and margins
% Replace `letterpaper' with`a4paper' for UK/EU standard size
\usepackage[a4paper,top=2cm,bottom=2cm,left=2cm,right=2cm,marginparwidth=1.75cm]{geometry}

\usepackage{amsmath}
\usepackage{graphicx}
\usepackage[colorlinks=true, allcolors=blue]{hyperref}
\usepackage{amsfonts}
\usepackage{amssymb}
% \usepackage[left=1cm,right=1cm,top=1cm,bottom=1cm]{geometry}
\usepackage{hyperref}
\usepackage{seqsplit}
\usepackage[dvipsnames]{xcolor}
\usepackage{enumitem}
\usepackage{algorithm}
\usepackage{algpseudocode}
\usepackage{algorithmicx}
\usepackage{mathalfa}
\usepackage{mathrsfs}
\usepackage{dsfont}
\usepackage{caption,subcaption}
\usepackage{wrapfig}
\usepackage[stable]{footmisc}
\usepackage{indentfirst}
\usepackage{rotating}
\usepackage{pdflscape}
\usepackage{tikz}

\usepackage{MnSymbol,wasysym}
\usepackage{minted}

\begin{document}

\begin{center}
\Large {Задание 13. Динамическое программирование.}
\end{center}

\bigskip

\textbf{1[2 + 3]} Фирма производит программное обеспечение для банкоматов разных стран мира. Банкомату нужно выдавать запрашиваемую клиентом сумму минимальным количеством купюр.
\begin{enumerate}
    \item Если у банкомата есть купюры номиналом 1, 2, 5, 10, 20, 50, а сумма — 71, то набор банкнот
будет 50+20+1. Постройте алгоритм, который будет решать задачу для данного
набора купюр и произвольной суммы, которая является входом задачи.
    \item Постройте алгоритм, который решает задачу, когда на вход помимо суммы подаются
и номиналы банкнот. Является ли он полиномиальным?
\end{enumerate}

\medskip

\textbf{2[2]} Постройте алгоритм, определяющий, содержит ли данный неориентированный граф (простой) цикл длины 4. Решите в двух случаях, когда имеется в виду реберная длина и вес пути.

\medskip

\textbf{3[2]} Как модифицировать алгоритм Флойда-Уоршелла, чтобы он находил не только длины кратчайших путей между всеми парами вершин, но и сами пути?

\medskip

\textbf{4[2 + 3]} Назовём последовательность $x_1, x_2, \dots, , x_n$ строго унимодальной, если существует такой индекс $k$, что $x_1 < x_2 < \dots < x_k > x_{k+1} > \dots > x_n$. Постройте алгоритм, который получает на вход конечную последовательность натуральных чисел и находит её самую длинную строго унимодальную подпоследовательность.

\end{document}