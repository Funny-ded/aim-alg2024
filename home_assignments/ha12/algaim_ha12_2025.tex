\documentclass{article}
\usepackage[T2A]{fontenc}
\usepackage[utf8]{inputenc}   
\usepackage[english, russian]{babel}

% Set page size and margins
% Replace `letterpaper' with`a4paper' for UK/EU standard size
\usepackage[a4paper,top=2cm,bottom=2cm,left=2cm,right=2cm,marginparwidth=1.75cm]{geometry}

\usepackage{amsmath}
\usepackage{graphicx}
\usepackage[colorlinks=true, allcolors=blue]{hyperref}
\usepackage{amsfonts}
\usepackage{amssymb}
% \usepackage[left=1cm,right=1cm,top=1cm,bottom=1cm]{geometry}
\usepackage{hyperref}
\usepackage{seqsplit}
\usepackage[dvipsnames]{xcolor}
\usepackage{enumitem}
\usepackage{algorithm}
\usepackage{algpseudocode}
\usepackage{algorithmicx}
\usepackage{mathalfa}
\usepackage{mathrsfs}
\usepackage{dsfont}
\usepackage{caption,subcaption}
\usepackage{wrapfig}
\usepackage[stable]{footmisc}
\usepackage{indentfirst}
\usepackage{rotating}
\usepackage{pdflscape}
\usepackage{tikz}

\usepackage{MnSymbol,wasysym}
\usepackage{minted}

\begin{document}

\begin{center}
\Large {Задание 12. Минимальные остовные деревья. Динамическое программирование.}
\end{center}

\bigskip

\textbf{1[1]} Постройте алгоритм, который находит максимальное остовное дерево графа, то есть
остовное дерево максимального веса.

\medskip

\textbf{2[1]} Верно ли, что дерево кратчайших путей, которое строит алгоритм Дейкстры, является минимальным
остовным деревом?

\medskip

\textbf{3[3]} На вход задачи подаётся неориентированный взвешенный граф $G(V, E)$ и подмножество вершин $U \subseteq V$. Необходимо построить остовное дерево, минимальное (по весу) среди деревьев, в которых все вершины $U$ являются листьями (но могут быть и другие листья) или обнаружить, что таких остовных деревьев нет. Постройте алгоритм, который решает задачу за $O(|E| \log |V |)$. Обратите внимание, что искомое дерево может не быть минимальным остовным деревом.

\medskip

\textbf{4[2 + 3]} Рассмотрим следующую игру. На доске нарисовано $n$ палочек. Два игрока по очереди зачёркивают от одной до трёх палочек. Проигрывает тот, кто зачеркнул последнюю палочку.

\begin{enumerate}
    \item Кто выигрывает при $n = 20$? (Считая, что соперник не ошибается.)
    \item Кто выигрывает при произвольном $n$?
\end{enumerate}

\textbf{5[3]} Два игрока играют в следующую игру. На поле из $(N + 1) \times (N + 1)$ клеток (нумерация от $0$ до $N$) в клетке $(0, 0)$ стоит фишка. Её разрешено разрешено двигать из клетки с координатами $(x, y)$ в клетку с координатами $(x + a_i, y + b_i)$, где пары неотрицательных целых чисел $(a_i, b_i)$ обговорены перед началом игры; при этом $a_i$ и $b_i$ не равны нулю одновременно. Выигрывает тот игрок, который первым вывел фишку в клетку, которая находится на расстоянии не менее чем $R$ от $(0, 0)$. Необходимо определить, кто из игроков выигрывает при безошибочных действиях соперника. Игроки ходят по очереди, пропускать ход нельзя.

\begin{enumerate}
    \item Определите, кто из игроков имеет выигрышную стратегию при безошибочной игре соперника,
если N = 5, R = 5, а список допустимых ходов: (1, 2), (2, 1), (1, 1).
    \item Постройте алгоритм, который определяет победителя и его выигрышную стратегию в общем
случае и оцените его сложность.
\end{enumerate}

\end{document}