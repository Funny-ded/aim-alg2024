\documentclass{article}
\usepackage[T2A]{fontenc}
\usepackage[utf8]{inputenc}   
\usepackage[english, russian]{babel}

% Set page size and margins
% Replace `letterpaper' with`a4paper' for UK/EU standard size
\usepackage[a4paper,top=2cm,bottom=2cm,left=2cm,right=2cm,marginparwidth=1.75cm]{geometry}

\usepackage{amsmath}
\usepackage{graphicx}
\usepackage[colorlinks=true, allcolors=blue]{hyperref}
\usepackage{amsfonts}
\usepackage{amssymb}
% \usepackage[left=1cm,right=1cm,top=1cm,bottom=1cm]{geometry}
\usepackage{hyperref}
\usepackage{seqsplit}
\usepackage[dvipsnames]{xcolor}
\usepackage{enumitem}
\usepackage{algorithm}
\usepackage{algpseudocode}
\usepackage{algorithmicx}
\usepackage{mathalfa}
\usepackage{mathrsfs}
\usepackage{dsfont}
\usepackage{caption,subcaption}
\usepackage{wrapfig}
\usepackage[stable]{footmisc}
\usepackage{indentfirst}
\usepackage{rotating}
\usepackage{pdflscape}
\usepackage{tikz}

\usepackage{MnSymbol,wasysym}
\usepackage{minted}

\begin{document}

\begin{center}
\Large {Задание 9. Кратчайшие пути в графах}
\end{center}

\bigskip

\textbf{1[2]} Примените алгоритм Беллмана-Форда к графу для поиска кратчайших путей от вершины $e$ до всех остальных

\begin{center}
\begin{tikzpicture}[scale=0.2]
\tikzstyle{every node}+=[inner sep=0pt]
\draw [black] (33.1,-23.4) circle (3);
\draw (33.1,-23.4) node {$b$};
\draw [black] (20.2,-23.4) circle (3);
\draw (20.2,-23.4) node {$e$};
\draw [black] (20.2,-36.5) circle (3);
\draw (20.2,-36.5) node {$h$};
\draw [black] (46.1,-23.4) circle (3);
\draw (46.1,-23.4) node {$a$};
\draw [black] (58.6,-23.4) circle (3);
\draw (58.6,-23.4) node {$d$};
\draw [black] (53.5,-36.5) circle (3);
\draw (53.5,-36.5) node {$f$};
\draw [black] (40.1,-36.5) circle (3);
\draw (40.1,-36.5) node {$g$};
\draw [black] (58.749,-26.39) arc (-3.31225:-39.23097:13.933);
\fill [black] (58.75,-26.39) -- (58.2,-27.16) -- (59.2,-27.22);
\draw (58.57,-31.5) node [right] {$-3$};
\draw [black] (53.391,-33.507) arc (-183.90437:-218.63884:14.338);
\fill [black] (53.39,-33.51) -- (53.94,-32.74) -- (52.95,-32.68);
\draw (53.58,-28.43) node [left] {$4$};
\draw [black] (35.795,-22.1) arc (108.5622:71.4378:11.953);
\fill [black] (43.41,-22.1) -- (42.81,-21.37) -- (42.49,-22.32);
\draw (39.6,-20.98) node [above] {$4$};
\draw [black] (23.2,-36.5) -- (37.1,-36.5);
\fill [black] (37.1,-36.5) -- (36.3,-36) -- (36.3,-37);
\draw (30.15,-36) node [above] {$-3$};
\draw [black] (22.71,-25.05) -- (37.59,-34.85);
\fill [black] (37.59,-34.85) -- (37.2,-33.99) -- (36.65,-34.83);
\draw (28.82,-30.45) node [below] {$-2$};
\draw [black] (43.1,-36.5) -- (50.5,-36.5);
\fill [black] (50.5,-36.5) -- (49.7,-36) -- (49.7,-37);
\draw (46.8,-37) node [below] {$1$};
\draw [black] (23.09,-22.608) arc (100.80296:79.19704:18.992);
\fill [black] (30.21,-22.61) -- (29.52,-21.97) -- (29.33,-22.95);
\draw (26.65,-21.77) node [above] {$-1$};
\draw [black] (51.438,-34.323) arc (-139.9749:-161.10215:25.165);
\fill [black] (46.9,-26.29) -- (46.69,-27.21) -- (47.63,-26.88);
\draw (48.14,-31.73) node [left] {$2$};
\end{tikzpicture}
\end{center}

\medskip

\textbf{2[3]} Пусть сильно связный граф, в котором кратчайшие расстояния между всеми парами вершин имеют реберную длину не более $k$, называется \textit{k-плотненьким}.

Предложите эффективный алгоритм поиска кратчайших путей в \textit{k-плотненьких} графах. Оцените его асимптотику.

\medskip

\textbf{3[3]} Независимое множество в неориентированном графе - это множество вершин, попарно не соединенных ребрами. Предложите $O(|V| + |E|)$ алгоритм поиска максимального по размеру независимого множества в дереве.

\medskip

\textbf{4[3]} Предложите $O(|V | + |E|)$ алгоритм поиска кратчайших расстояний от данной вершины $s$ до всех остальных в графе с весами ребер $0$ или $1$. Докажите его корректность и оцените асимптотику.

%\medskip

%\textbf{5[1 + 5]} 

%\medskip

%\textbf{6[2]} 

%\medskip

%\textbf{7[3]} 

%\medskip

%\textbf{8[2]} 

\end{document}