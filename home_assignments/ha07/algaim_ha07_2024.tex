\documentclass{article}
\usepackage[T2A]{fontenc}
\usepackage[utf8]{inputenc}   
\usepackage[english, russian]{babel}

% Set page size and margins
% Replace `letterpaper' with`a4paper' for UK/EU standard size
\usepackage[a4paper,top=2cm,bottom=2cm,left=2cm,right=2cm,marginparwidth=1.75cm]{geometry}

\usepackage{amsmath}
\usepackage{graphicx}
\usepackage[colorlinks=true, allcolors=blue]{hyperref}
\usepackage{amsfonts}
\usepackage{amssymb}
% \usepackage[left=1cm,right=1cm,top=1cm,bottom=1cm]{geometry}
\usepackage{hyperref}
\usepackage{seqsplit}
\usepackage[dvipsnames]{xcolor}
\usepackage{enumitem}
\usepackage{algorithm}
\usepackage{algpseudocode}
\usepackage{algorithmicx}
\usepackage{mathalfa}
\usepackage{mathrsfs}
\usepackage{dsfont}
\usepackage{caption,subcaption}
\usepackage{wrapfig}
\usepackage[stable]{footmisc}
\usepackage{indentfirst}
\usepackage{rotating}
\usepackage{pdflscape}
\usepackage{tikz}

\usepackage{MnSymbol,wasysym}
\usepackage{minted}

\begin{document}

\begin{center}
\Large {Задание 7. Расширенный алгоритм Евклида. Куча. Разные задачи. Графы}
\end{center}

\bigskip

\textbf{1} Решите уравнения в целых числах. Нужно найти все решения, а не только частное.

\begin{enumerate}
    \item $238x + 385y = 133$
    \item $143x + 121y = 52$
\end{enumerate}

\medskip

\textbf{2} Решите сравнение $68x + 85 \equiv 0 (\mod 561)$ с помощью расширенного алгоритма Евклида. Требуется найти все решения в вычетах.

\medskip

\textbf{3} Найдите обратный остаток $7^{-1} (\mod 102)$

\medskip

\textbf{4} Приведите алгоритм добавления элемента в уже существующую кучу на максимум из $n$ элементов. Докажите его корректность и оцените асимптотику.

\medskip

\textbf{5} Реализуйте очередь через два стека. Оцените асимптотику операций с получившейся очередью.

\medskip

\textbf{6} Сколько существует различных лесов обхода в глубину для графа-пути?

\medskip

\textbf{7} Турнир - это полный ориентированный граф, то есть такой ориентированный граф, в котором между любой парой различных вершин есть ровно одно ребро. Докажите, что в турнире на $n$ вершинах есть простой (несамопересекающийся) путь длины $n - 1$. Постройте алгоритм, находящий такой путь, и оцените время его работы.

\medskip

\textbf{8} Примените DFS к графу.

\begin{center}
\begin{tikzpicture}[scale=0.2]
\tikzstyle{every node}+=[inner sep=0pt]
\draw [black] (19.4,-16.8) circle (3);
\draw (19.4,-16.8) node {$a$};
\draw [black] (39.2,-16.6) circle (3);
\draw (39.2,-16.6) node {$b$};
\draw [black] (57.9,-16.6) circle (3);
\draw (57.9,-16.6) node {$c$};
\draw [black] (18.9,-36.4) circle (3);
\draw (18.9,-36.4) node {$d$};
\draw [black] (39.2,-36.4) circle (3);
\draw (39.2,-36.4) node {$e$};
\draw [black] (58.5,-36.4) circle (3);
\draw (58.5,-36.4) node {$f$};
\draw [black] (22.4,-16.77) -- (36.2,-16.63);
\fill [black] (36.2,-16.63) -- (35.4,-16.14) -- (35.41,-17.14);
\draw [black] (18.9,-19.8) -- (18.9,-33.4);
\fill [black] (18.9,-33.4) -- (19.4,-32.6) -- (18.4,-32.6);
\draw [black] (39.2,-19.6) -- (39.2,-33.4);
\fill [black] (39.2,-33.4) -- (39.7,-32.6) -- (38.7,-32.6);
\draw [black] (42.2,-16.6) -- (54.9,-16.6);
\fill [black] (54.9,-16.6) -- (54.1,-16.1) -- (54.1,-17.1);
\draw [black] (42.2,-36.4) -- (55.5,-36.4);
\fill [black] (55.5,-36.4) -- (54.7,-35.9) -- (54.7,-36.9);
\draw [black] (57.99,-19.6) -- (58.41,-33.4);
\fill [black] (58.41,-33.4) -- (58.88,-32.59) -- (57.89,-32.62);
\draw [black] (36.2,-36.4) -- (21.9,-36.4);
\fill [black] (21.9,-36.4) -- (22.7,-36.9) -- (22.7,-35.9);
\end{tikzpicture}
\end{center}

Порядок выбора вершин алфавитный.

\medskip

\textbf{9$^*$} На вход задачи подаются натуральные числа $n, a_0, \dots, a_n, y$. Необходимо определить, существует ли такое натуральное число $x$, что $y = a_n x^n + a_{n-1} x^{n-1} + \dots + a_1 x + a_0$.

\medskip

\textbf{10$^*$} Ваш лектор по алгоритмам нашёл два одинаковых шарика из неизвестного материала и внезапно решил измерить их прочность в этажах стоэтажного небоскрёба. Прочность равна номеру минимального этажа, при броске шарика из окна которого шарик
разобъётся (максимум 100). Считаем, что если шарик уцелел, то его прочность после броска не уменьшилась. Сколько бросков шариков необходимо и достаточно для нахождения прочности?

\end{document}

%На вход поступает ориентированный ациклический граф $G = (V, E)$ и две его вершины $s$ и $t$.

%Построим граф следующим образом. Возьмем граф-путь на $n$ вершинах (ориентированный граф, в котором есть ребро из первой вершины во вторую, из второй в третью и так далее) и добавим к нему еще одну вершину. Добавим ребра из всех вершин исходного графа в добавленную. Сколько существует различных лесов обхода в глубину для такого графа?

Возьмем два графа-кольца (ребра идут из $i$-й вершины в $i+1$-ю и одно из $n$-й в первую) на $n$ вершинах и соединим две произвольные вершины разных колец ребром. Сколько существует различных лесов обхода в глубину в таком графе?