\documentclass{article}
\usepackage[T2A]{fontenc}
\usepackage[utf8]{inputenc}   
\usepackage[english, russian]{babel}

% Set page size and margins
% Replace `letterpaper' with`a4paper' for UK/EU standard size
\usepackage[a4paper,top=2cm,bottom=2cm,left=2cm,right=2cm,marginparwidth=1.75cm]{geometry}

\usepackage{amsmath}
\usepackage{graphicx}
\usepackage[colorlinks=true, allcolors=blue]{hyperref}
\usepackage{amsfonts}
\usepackage{amssymb}
% \usepackage[left=1cm,right=1cm,top=1cm,bottom=1cm]{geometry}
\usepackage{hyperref}
\usepackage{seqsplit}
\usepackage[dvipsnames]{xcolor}
\usepackage{enumitem}
\usepackage{algorithm}
\usepackage{algpseudocode}
\usepackage{algorithmicx}
\usepackage{mathalfa}
\usepackage{mathrsfs}
\usepackage{dsfont}
\usepackage{caption,subcaption}
\usepackage{wrapfig}
\usepackage[stable]{footmisc}
\usepackage{indentfirst}
\usepackage{rotating}
\usepackage{pdflscape}

\usepackage{MnSymbol,wasysym}
\usepackage{minted}

\begin{document}

\begin{center}
\Large {Задание 3. Сортировки, рекурренты.}
\end{center}

\bigskip

\textbf{1} К серверу приходят одновременно $n$ клиентов.
Для клиента $i$ известно время его обслуживания $t_i$.
Время ожидания клиента определяется как сумма времени обслуживания всех предыдущих клиентов и времени обслуживания его самого.
К примеру, если обслуживает клиентов в порядке номеров, то время ожидания клиента i будет равно $\sum_{j=1}^i t_j$.
Постройте эффективный алгоритм, находящий последовательность обслуживания клиентов с минимальным суммарным временем ожидания клиентов, докажите его корректность и оцените асимптотику.

\medskip

\textbf{2} Найдите асимптотическую оценку функции $T(n)$. Примените мастер-теорему в тех случаях, когда ее можно использовать.

\begin{enumerate}
    \item $T(n) = 25 T(\frac{n}{5}) + n^2$
    \item $T(n) = 16 T(\frac{n}{2}) + n^3$
    \item $T(n) = 9 T(\frac{n}{3}) + n^3$
    \item $T(n) = T(n-1) + 3n$
    \item $T(n) = T(\frac{n}{4}) + T(\frac{3n}{4}) + n$
\end{enumerate}

\medskip

\textbf{3} Оцените асимптотически, сколько раз будет напечатана строка ''heh'' при вызове функции $\text{f}$.

\begin{minted}{python}
    def f(n):
        f(n // 2)
        
        for i = 1; i < n; i += 2:
            print("heh")

        f(n // 2)
        f(n // 2)
\end{minted}

\end{document}