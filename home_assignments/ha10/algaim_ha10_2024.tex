\documentclass{article}
\usepackage[T2A]{fontenc}
\usepackage[utf8]{inputenc}   
\usepackage[english, russian]{babel}

% Set page size and margins
% Replace `letterpaper' with`a4paper' for UK/EU standard size
\usepackage[a4paper,top=2cm,bottom=2cm,left=2cm,right=2cm,marginparwidth=1.75cm]{geometry}

\usepackage{amsmath}
\usepackage{graphicx}
\usepackage[colorlinks=true, allcolors=blue]{hyperref}
\usepackage{amsfonts}
\usepackage{amssymb}
% \usepackage[left=1cm,right=1cm,top=1cm,bottom=1cm]{geometry}
\usepackage{hyperref}
\usepackage{seqsplit}
\usepackage[dvipsnames]{xcolor}
\usepackage{enumitem}
\usepackage{algorithm}
\usepackage{algpseudocode}
\usepackage{algorithmicx}
\usepackage{mathalfa}
\usepackage{mathrsfs}
\usepackage{dsfont}
\usepackage{caption,subcaption}
\usepackage{wrapfig}
\usepackage[stable]{footmisc}
\usepackage{indentfirst}
\usepackage{rotating}
\usepackage{pdflscape}
\usepackage{tikz}

\usepackage{MnSymbol,wasysym}
\usepackage{minted}

\begin{document}

\begin{center}
\Large {Задание 10. Доказательства утверждений в графах. Кратчайшие пути. Разные задачи.}
\end{center}

\bigskip

\textit{Для доказательства корректности алгоритма нужно предоставить набор утверждений, логически следующих из условия задачи и описания алгоритма и из которых в свою очередь следует, что на всех возможных входах алгоритм даёт верный ответ.}

\medskip

\textbf{1[2]} Кастелянша ходит по общежитию и производит устное внушение о последствиях ненадлежащего поведения живущих там студентов. Между двумя комнатами или можно пройти напрямую, или нельзя (в таком случае можно пройти через промежуточные комнаты). В общежитии $|V|$ комнат и $|E|$ коридоров между ними. Время прохождения каждого коридора известно и положительно. У одной комнаты может быть не более $k$ коридоров. Оказавшись в комнате, кастелянша всегда производит воспитательную работу, время которой составляет 3 минуты. Постройте алгоритм, позволяющий кастелянше по заданной конфигурации общежития быстро узнать, за какое минимальное время она может дойти от своего кабинета до заданной комнаты, докажите его корректность и оцените асимптотику.

\medskip

\textbf{2[0 + 8]} В государстве между $n$ городами есть $m$ одностронних дорог. Было решено разделить города государства на наименьшее количество областей так, чтобы внутри каждой области
все города были достижимы друг из друга.

1. Предложите эффективный алгоритм, который осуществляет такое разделение, докажите его корректность и оцените асимптотику.

2. Государство решило добиться того, чтобы из каждого города можно было добраться до каждого. В силу бюджетных ограничений было решено построить минимальное число односторонних дорог (любой длины), необходимое для достижения этой цели. Предложите алгоритм, решающий задачу.

\medskip

\textbf{3[2]} Предложите алгоритм, выясняющий, есть ли в неориентированном графе циклы нечетной длины. Оцените асимптотику его работы. Докажите его корректность.

\medskip

\textbf{4[2]} Диаметр графа - это максимальное кратчайшее расстояниие между двумя его вершинами. Рассмотрим граф состояний кодового замка с 4 разрядами. Каждая комбинация соответствует одной вершине. Ребрами соединены комбинации, между которыми возможен переход вращением одного из колесиков на единицу вверх или вниз. Все ребра имеют единичный вес. Найдите диаметр этого графа.

\medskip

\textbf{5[3]} В маленьком городе с населением в $n$ человек все всех знают. Это означает, что каждый человек знаком с каждым через не более чем $k$ других. В городе живёт $m$ людей с таксами. Поскольку у такс короткие ноги, они ходят медленно, и их хозяева не успели познакомиться между собой. Предложите алгоритм, который позволит каждому из них найти кратчайшую цепочку знакомств, ведущую к другому хозяину таксы.

%\medskip

%\textbf{5[1 + 5]} 

%\medskip

%\textbf{6[2]} 

%\medskip

%\textbf{7[3]} 

%\medskip

%\textbf{8[2]} 

\end{document}