\documentclass{article}
\usepackage[T2A]{fontenc}
\usepackage[utf8]{inputenc}   
\usepackage[english, russian]{babel}

% Set page size and margins
% Replace `letterpaper' with`a4paper' for UK/EU standard size
\usepackage[a4paper,top=2cm,bottom=2cm,left=2cm,right=2cm,marginparwidth=1.75cm]{geometry}

\usepackage{amsmath}
\usepackage{graphicx}
\usepackage[colorlinks=true, allcolors=blue]{hyperref}
\usepackage{amsfonts}
\usepackage{amssymb}
% \usepackage[left=1cm,right=1cm,top=1cm,bottom=1cm]{geometry}
\usepackage{hyperref}
\usepackage{seqsplit}
\usepackage[dvipsnames]{xcolor}
\usepackage{enumitem}
\usepackage{algorithm}
\usepackage{algpseudocode}
\usepackage{algorithmicx}
\usepackage{mathalfa}
\usepackage{mathrsfs}
\usepackage{dsfont}
\usepackage{caption,subcaption}
\usepackage{wrapfig}
\usepackage[stable]{footmisc}
\usepackage{indentfirst}
\usepackage{rotating}
\usepackage{pdflscape}
\usepackage{tikz}

\usepackage{MnSymbol,wasysym}
\usepackage{minted}

\begin{document}

\begin{center}
\Large {Задание 14. Разные задачи.}
\end{center}

\bigskip

\textbf{1[3]} Профессор О. П. Рометчивый придумал новый тип нейронных сетей. На вход поступает вектор $x$ размерности $n_1$. Он умножается слева на матрицу $A_1 \in \mathbb{R}^{n_2 \times n_1}$ ($A_1 x$). Затем результат умножается на матрицу $A_2$, $A_3$ и так далее. Всего профессор использует $m$ матриц, получая вектор размерности $n_{m+1}$, и он написал в своей статье значения $n_1, n_2, \dots, n_{m+1}$. Сколько операций умножения и сложения потребуется сделать, чтобы получить выход такой сети на одном входном векторе? Как оптимизировать вычисление выхода? Какая сложность будет у оптимизированного алгоритма?

\medskip

\textbf{2[2]} На вход задачи подаётся число n и последовательность целых чисел $a_1, \dots, a_n$. Необходимо
найти такие номера $i$ и $j (1 \leq i < j \leq n)$, что сумма $\sum_{k=i}^{j} a_k$ максимальна.
Постройте линейный алгоритм, решающий задачу.


\end{document}