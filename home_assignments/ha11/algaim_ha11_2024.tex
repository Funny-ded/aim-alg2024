\documentclass{article}
\usepackage[T2A]{fontenc}
\usepackage[utf8]{inputenc}   
\usepackage[english, russian]{babel}

% Set page size and margins
% Replace `letterpaper' with`a4paper' for UK/EU standard size
\usepackage[a4paper,top=2cm,bottom=2cm,left=2cm,right=2cm,marginparwidth=1.75cm]{geometry}

\usepackage{amsmath}
\usepackage{graphicx}
\usepackage[colorlinks=true, allcolors=blue]{hyperref}
\usepackage{amsfonts}
\usepackage{amssymb}
% \usepackage[left=1cm,right=1cm,top=1cm,bottom=1cm]{geometry}
\usepackage{hyperref}
\usepackage{seqsplit}
\usepackage[dvipsnames]{xcolor}
\usepackage{enumitem}
\usepackage{algorithm}
\usepackage{algpseudocode}
\usepackage{algorithmicx}
\usepackage{mathalfa}
\usepackage{mathrsfs}
\usepackage{dsfont}
\usepackage{caption,subcaption}
\usepackage{wrapfig}
\usepackage[stable]{footmisc}
\usepackage{indentfirst}
\usepackage{rotating}
\usepackage{pdflscape}
\usepackage{tikz}

\usepackage{MnSymbol,wasysym}
\usepackage{minted}

\begin{document}

\begin{center}
\Large {Задание 11. Кратчайшие пути. Разные задачи.}
\end{center}

\bigskip

\textit{Для доказательства корректности алгоритма нужно предоставить набор утверждений, логически следующих из условия задачи и описания алгоритма и из которых в свою очередь следует, что на всех возможных входах алгоритм даёт верный ответ.}

\medskip

\textbf{1[2]} Все степени вершин в неориентированном графе равны $2k$. Все его ребра покрашены в несколько цветов. Предложите $O(V + E)$ алгоритм, находящий в этом графе эйлеров цикл, в котором цвета всех соседних ребер различны, либо выводящий, что такого цикла нет.

\medskip

\textbf{2[5]} Веса ребер графа лежат в интервале $0, 1, \dots, W$, где $W$ - константа. Предложите алгоритм поиска кратчайших путей от одной вершины до всех остальных с временем работы $O(W|V| + |E|)$.

\medskip

\textbf{3[2+3+3]} Рассмотрим следующую модификацию алгоритма Дейкстры. При инициализации в очереди с приоритетами находится лишь вершина $s$. Вершина $v$ добавляется в очередь с приоритетами, если в результате релаксации $Relax(u, v)$ расстояние до вершины $v$ изменилось, и при этом $v$ не была в этот момент в очереди. Остальные шаги алгоритма остаются без изменений.

\begin{enumerate}
    \item Докажите корректность модифицированного алгоритма
    \item Докажите, что модифицированный алгоритм работает корректно даже при наличии рёбер отрицательного веса, но при отсутсвии циклов отрицательного веса. Оцените время работы алгоритма на графах такого вида и сравните его со временем работы алгоритма Беллмана-Форда
    \item Модифицируйте алгоритм так, чтобы он выдавал ошибку на графах с циклами отрицательного веса
\end{enumerate}

%\medskip

%\textbf{4[3]} 

%\medskip

%\textbf{5[2]} 

%\medskip

%\textbf{5[1 + 5]} 

%\medskip

%\textbf{6[2]} 

%\medskip

%\textbf{7[3]} 

%\medskip

%\textbf{8[2]} 

\end{document}