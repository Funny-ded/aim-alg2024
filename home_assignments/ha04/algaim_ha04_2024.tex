\documentclass{article}
\usepackage[T2A]{fontenc}
\usepackage[utf8]{inputenc}   
\usepackage[english, russian]{babel}

% Set page size and margins
% Replace `letterpaper' with`a4paper' for UK/EU standard size
\usepackage[a4paper,top=2cm,bottom=2cm,left=2cm,right=2cm,marginparwidth=1.75cm]{geometry}

\usepackage{amsmath}
\usepackage{graphicx}
\usepackage[colorlinks=true, allcolors=blue]{hyperref}
\usepackage{amsfonts}
\usepackage{amssymb}
% \usepackage[left=1cm,right=1cm,top=1cm,bottom=1cm]{geometry}
\usepackage{hyperref}
\usepackage{seqsplit}
\usepackage[dvipsnames]{xcolor}
\usepackage{enumitem}
\usepackage{algorithm}
\usepackage{algpseudocode}
\usepackage{algorithmicx}
\usepackage{mathalfa}
\usepackage{mathrsfs}
\usepackage{dsfont}
\usepackage{caption,subcaption}
\usepackage{wrapfig}
\usepackage[stable]{footmisc}
\usepackage{indentfirst}
\usepackage{rotating}
\usepackage{pdflscape}

\usepackage{MnSymbol,wasysym}
\usepackage{minted}

\begin{document}

\begin{center}
\Large {Задание 4. Сортировки, порядковые статистики.}
\end{center}

\bigskip

\textbf{Указание:} в этом задании мы будем использовать алгоритм поиска $k$-й порядковой статистики в качестве черного ящика, принимающего массив и число $k$, и за линейное время находящего элемент, который будет стоять на $k$-м месте в отсортированном массиве.

\medskip

\textbf{Рекомендация:} докажите корректность и оцените асимптотику для каждой задачи.

\medskip

\textbf{1} На вход задачи поступают три отсортированных массива. Постройте алгоритм, находящий число уникальных элементов в объединении этих массивов.

\medskip

\textbf{2} На вход задачи поступает массив $a$ из $n$ чисел. Постройте алгоритм, находящий число инверсий в массиве, то есть таких пар индексов $i, j$, что $i < j$ и $a[i] > a[j]$.

\textbf{Рекомендация:} модифицируйте алгоритм сортировки слиянием.

\medskip

\textbf{3} На прямой задано $n$ строго вложенных отрезков в виде пар концов $(l_i, r_i)$. Они могут поступать на вход в произвольном порядке. Постройте алгоритм, находящий \textbf{множество точек на прямой}, покрытое ровно $\lceil \frac{2n}{3} \rceil$ отрезками.

\medskip

\textbf{4} На вход поступает число $n$ и массив $a$ размера $2n + 1$. Постройте алгоритм, находящий число $s$, минимизирующее сумму $\sum\limits_{i=1}^{2n+1} |a_i - s|$

\end{document}