\documentclass{article}
\usepackage[T2A]{fontenc}
\usepackage[utf8]{inputenc}   
\usepackage[english, russian]{babel}

% Set page size and margins
% Replace `letterpaper' with`a4paper' for UK/EU standard size
\usepackage[a4paper,top=2cm,bottom=2cm,left=2cm,right=2cm,marginparwidth=1.75cm]{geometry}

\usepackage{amsmath}
\usepackage{graphicx}
\usepackage[colorlinks=true, allcolors=blue]{hyperref}
\usepackage{amsfonts}
\usepackage{amssymb}
% \usepackage[left=1cm,right=1cm,top=1cm,bottom=1cm]{geometry}
\usepackage{hyperref}
\usepackage{seqsplit}
\usepackage[dvipsnames]{xcolor}
\usepackage{enumitem}
\usepackage{algorithm}
\usepackage{algpseudocode}
\usepackage{algorithmicx}
\usepackage{mathalfa}
\usepackage{mathrsfs}
\usepackage{dsfont}
\usepackage{caption,subcaption}
\usepackage{wrapfig}
\usepackage[stable]{footmisc}
\usepackage{indentfirst}
\usepackage{rotating}
\usepackage{pdflscape}

\usepackage{MnSymbol,wasysym}
\usepackage{minted}

\begin{document}

\begin{center}
\Large {Задание 5. Статистики, битовые операции, нижние оценки.}
\end{center}

\bigskip

\textbf{1} На вход подается массив $a_1, \cdots, a_n$, в котором один из элементов встречается не меньше $\lceil \frac{n}{2} \rceil$ раз. Постройте алгоритм, находящий этот элемент.

\medskip

\textbf{2} С какой асимптотикой будеет работать алгоритм поиска $k$-й порядковой статистики, если делить массив на группы не по пять элементов, а по три? По семь?

\medskip

\textbf{3} На доске написан набор положительных целых чисел. За один ход можно взять любые два числа и вычесть из большего меньшее. Процесс останавливается, когда все числа становятся одинаковыми. Докажите, что этот процесс всегда остановится. Какие числа останутся в результате?

\medskip

\textbf{4} Оцените сложность алгоритма Divide, приведенного на странице 19 книги Дасгупты.

\medskip

\textbf{5} Предложите алгоритм возведения $n$-битовых чисел в степень по модулю, оцените его сложность.

\medskip

\textbf{6} Вам дана кучка камней, один из которых радиоактивный. Счетчиком Гейгера можно измерить, есть ли в любой кучке радиоактивный. Предложите алгоритм, позволяющий найти этот камень за асимптотически оптимальное число действий. Докажите, что меньше действий совершить нельзя.

% \medskip

% \textbf{7} 

% \medskip

% \textbf{8} 

% \medskip

% \textbf{9} 

% \medskip

% \textbf{10} 

\end{document}