\documentclass{article}
\usepackage[T2A]{fontenc}
\usepackage[utf8]{inputenc}   
\usepackage[english, russian]{babel}

% Set page size and margins
% Replace `letterpaper' with`a4paper' for UK/EU standard size
\usepackage[a4paper,top=2cm,bottom=2cm,left=2cm,right=2cm,marginparwidth=1.75cm]{geometry}

\usepackage{amsmath}
\usepackage{graphicx}
\usepackage[colorlinks=true, allcolors=blue]{hyperref}
\usepackage{amsfonts}
\usepackage{amssymb}
% \usepackage[left=1cm,right=1cm,top=1cm,bottom=1cm]{geometry}
\usepackage{hyperref}
\usepackage{seqsplit}
\usepackage[dvipsnames]{xcolor}
\usepackage{enumitem}
\usepackage{algorithm}
\usepackage{algpseudocode}
\usepackage{algorithmicx}
\usepackage{mathalfa}
\usepackage{mathrsfs}
\usepackage{dsfont}
\usepackage{caption,subcaption}
\usepackage{wrapfig}
\usepackage[stable]{footmisc}
\usepackage{indentfirst}
\usepackage{rotating}
\usepackage{pdflscape}

\usepackage{MnSymbol,wasysym}
\usepackage{minted}

\begin{document}

\begin{center}
\Large {Задание 6. Разные задачи. Расширенный алгоритм Евклида}
\end{center}

\bigskip

\textbf{1} Оцените асимптотику роста функции $f(n) = 1 + c + c^2 + \dots + c^n$ в зависимости от параметра $c$.

\medskip

\textbf{2} Покажите, что для рекурренты $T(n) = T(n - 1) + 4 T(n - 3)$ верна оценка $\log T(n) = \Theta(n)$.

\medskip

\textbf{3} Предложите эффективный алгоритм вычисления наименьшего общего кратного (НОК) двух чисел в битовой модели вычислений (время выполнения операций зависит от длины битовой записи чисел).

\medskip

\textbf{4} Найдите тета-асимптотику для рекурренты $T(n) = 3 T(\frac{n}{4}) + T(\frac{n}{6}) + n$.

\medskip

\textbf{5} Найдите представление НОД чисел $a = 36$ и $b = 45$ в виде их линейной комбинации, то есть таких чисел $x$ и $y$, что $ax + by = gcd(a, b)$. Воспользуйтесь расширенным алгоритмом Евклида для решения этой задачи.

\medskip

\textbf{6} [Задача о Ханойской башне]. Есть три стержня, на первый из которых нанизано $n$ колец разного радиуса. Чем ниже лежит кольцо, тем больше радиус. Кольца разрешено перекладывать со стержня на стержень, но только при условии, что кольцо меньшего радиуса кладётся на кольцо
большего радиуса. Найдите минимальное число перекладываний, требуемое для того, чтобы переложить все кольца с одного стержня на другой, то есть оценку снизу для задачи, и приведите алгоритм, который ее достигает.

\medskip

\textbf{7} Дан массив из $n$ чисел. Нужно разбить этот массив на максимальное количество непрерывных подмассивов так, чтобы после сортировки элементов внутри каждого подмассива весь массив стал отсортированным. Предложите $O(n \log n)$ алгоритм для решения этой задачи.

% \medskip

% \textbf{8} 

% \medskip

% \textbf{9} 

% \medskip

% \textbf{10} 

\end{document}