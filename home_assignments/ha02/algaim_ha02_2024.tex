\documentclass{article}
\usepackage[T2A]{fontenc}
\usepackage[utf8]{inputenc}   
\usepackage[english, russian]{babel}

% Set page size and margins
% Replace `letterpaper' with`a4paper' for UK/EU standard size
\usepackage[a4paper,top=2cm,bottom=2cm,left=2cm,right=2cm,marginparwidth=1.75cm]{geometry}

\usepackage{amsmath}
\usepackage{graphicx}
\usepackage[colorlinks=true, allcolors=blue]{hyperref}
\usepackage{amsfonts}
\usepackage{amssymb}
% \usepackage[left=1cm,right=1cm,top=1cm,bottom=1cm]{geometry}
\usepackage{hyperref}
\usepackage{seqsplit}
\usepackage[dvipsnames]{xcolor}
\usepackage{enumitem}
\usepackage{algorithm}
\usepackage{algpseudocode}
\usepackage{algorithmicx}
\usepackage{mathalfa}
\usepackage{mathrsfs}
\usepackage{dsfont}
\usepackage{caption,subcaption}
\usepackage{wrapfig}
\usepackage[stable]{footmisc}
\usepackage{indentfirst}
\usepackage{rotating}
\usepackage{pdflscape}

\usepackage{MnSymbol,wasysym}
\usepackage{minted}

\begin{document}

\begin{center}
\Large {Задание 2. Cложности, быстрое возведение, Фибоначчи.}
\end{center}

\bigskip

\textbf{1} Найдите $7^{13} \mod 167$ с помощью быстрого возведения в степень. Нужно привести последовательность умножений и промежуточные результаты.

\medskip

\textbf{2} Злодей Анти-человек придумал последовательность чисел Анти-начи. Она продолжает последовательность чисел Фибоначчи влево. Например, поскольку $F_0 = 0$, $F_1 = 1$ и $F_{k+2} = F_{k+1} + F_k$, выполняется равенство $1 = 0 + F_{-1}$, из чего следует, что $F_{-1} = 1$. Дальнейшие числа Анти-начи определяются аналогично.

Анти-человек умеет быстро возводить матрицы в степень. Подскажите, как ему находить $F_k$ для отрицательных $k$.

\medskip

\textbf{3} На вход поступает $n$ котов целочисленной массы от $2$ до $k$ килограммов.
Для каждого кота известна масса и кличка.
Известно, что сначала накормить требуется наиболее худосочных.
Предложите алгоритм, выводящий порядок, в котором нужно кормить котов, докажите его корректность и оцените асимптотику.

\medskip

\textbf{4} Найдите $\Theta$-асимптотику функции $f(n) = \sum^n_{k=0} {n \choose k}$

\medskip

\textbf{5} Оцените асимптотически, сколько раз будет напечатана строка ''heh'' при вызове функции $\text{f}$.

\begin{minted}{python}
    def f(n):
        for i = 1; i < n; i *= 2:
            for j in range(i*i*i):
                print("heh")

        for i = 1; i < n; i += 2:
            for j in range(i*i):
                print("heh")
\end{minted}

\end{document}